% !TEX TS-program = pdflatexmk
% !TEX useTabsWithFiles)
% !TEX tabbedFile{./content/Abschnitt1.tex}(Abschnitt1)
% !TEX tabbedFile{./content/Abschnitt2.tex}(Abschnitt2)
% !TEX tabbedFile{./content/Abschnitt3.tex}(Test)
%% %%%%%%%%%%%%%%%%%%%%%%%%%%%%%%%%%%%%%%%%%%%%%%%%%%%%%%%%%%%%%%%
%%  Thema:  agfa-readme.tex   
% % Stand:	2022/03/08     
%% %%%%%%%%%%%%%%%%%%%%%%%%%%%%%%%%%%%%%%%%%%%%%%%%%%%%%%%%%%%%%%%
\documentclass[%	
	,toc		= bib 			 
	,parskip		= half-		 
	,headings	= normal			 	 		 
	,numbers		= noenddot		
	,leqno
	,version 	= last
% 	,DIV 		= calc
	,titlepage	= true
%	,twoside = true
	,ngerman				% Deutscher Text
%	,english				% Englischer Text	
	]{scrartcl}			% KOMA Artikelmodus
%% %%%%%%%%%%%%%%%%%%%%%%%%%%%%%%%%%%%%%%%%%%%%%%%%%%%%%%%%%%%%%%%

%% -- Hierin sind alle weiteren Pakete enthalten
%% -- Siehe agfa-readme.pdf für Details
%% -- 
 
\usepackage[%
%	,thmframed			% gerahmte Umgebungen
%	,lmodern				% lmodern
% 	,libertinus			% libertinus
%	,urldoi
	,numeric
	]{agfa-art} 

%% -- Was soll aktuell bearbeitet werden
%% -- Mittels % entsprechend steuern

\includeonly{%
	./content/Abschnitt0,
	./content/Abschnitt1,
	./content/Abschnitt2,
	./content/Abschnitt3
	} 

%% -- Für den Druck % entfernen; 
%% -- Notwendige Bindekorrektur

\KOMAoptions{BCOR = 12mm}

%% -- für die finale Version die folgenden Zeilen 
%% -- mit % auskommentieren

\KOMAoptions{footsepline}
\lofoot{UG}
\cofoot{Version}
\rofoot{\today}

%% -- Datei mit den Referenzen für Literatur
%% -- und Steuerung für url etc. 
%% -- siehe AGFA-ReadMe.pdf

%\addbibresource{./bib/agfa-bib.bib}	
\addbibresource{Buecher.bib}
\addbibresource{Artikel.bib}
\addbibresource{TeX.bib}
\addbibresource{CTAN.bib}

\ExecuteBibliographyOptions{%
	,backref		= true		% wo habe ich was referenziert, alles andere weg
 	,url		= true		% true	wenn online-Zitate separat gezeigt werden sollen
 	,doi		= false		% dito falls unerwünscht, sinnvoll bei Vorlesungen 
	,eprint		= false		% dito
	} 

%% -- Querverweise
%% -- siehe AGFA-ReadMe.pdf

\hypersetup{
	,colorlinks	= true    			%  Farbige Links false/true, für onlineversion true                                                           
	,urlcolor	= blue       		%                                                              
	,citecolor	= blue      		                                                              
	,linkcolor	= blue			 	% oder black
	,breaklinks = true
%  	,hidelinks 						% Vor dem Druck % entfernen
	}

%% -- Für das testen

\usepackage{ablatt-listings}

\renewcommand{\dictumwidth}{0.45\textwidth}

%% %%%%%%%%%%%%%%%%%%%%%%%%%%%%%%
\begin{document}
%% %%%%%%%%%%%%%%%%%%%%%%%%%
%% Frontmatter
%% %%%%%%%%%%%%%%%%%%%%%%%%
%% -- kann man anfangs übergehen
%% -- daher %% hier und bei \end{comment} entfernen 

%%\begin{comment}
\begingroup
	% !TEX root = ../agfa-readme.tex
%% Titelseite
%%
\begin{titlepage}
\titlehead{Universität Tübingen \\ 
	Mathematisch-Naturwissenschaftliche Fakultät \\ 
	Fachbereich Mathematik}
\subject{-- \LaTeX --}
\title{-- Die AGFA--Vorlagen --}
\subtitle{-- Eine Hilfestellung --}
\author{Ulrich Groh}
\date{\today} 
\publishers{}
\end{titlepage}
\maketitle
\cleardoublepage			% Titelseite
	\pagestyle{empty}
%	% !TEX root = ../agfa-readme.tex
%% %%%%%%%%%%%%%%%%%%%%%%%%%%%%%%%
\thispagestyle{empty} 
\section*{}
\cleardoubleoddpage
			% Danksagung
	\tableofcontents 
	\thispagestyle{empty}
\endgroup

%%% -- Zusammenfassung des Inhaltes
%
\cleardoubleoddpage 
	\pagenumbering{roman}
	% !TEX root = ../agfa-readme.tex
%% %%%%%%%%%%%%%%%%%%%%%%%%%%%%%%%
%% Zusammenfassung zu ../agfa-readme.tex
%% Stand: 2022/10/10
%% %%%%%%%%%%%%%%%%%%%%%%%%%%%%%%%
\thispagestyle{empty}
\dictum[G.\,W.\,F. Hegel]{Die Wahrheit einer Absicht ist die Tat.}
\addsec{Was beinhaltet die Vorlage}
Dies ist eine Übersicht zu der Vorlage, die ich für AGFA erstellt habe und die den Zweck hat, die Anfertigung einer Arbeit, sei es nun eine Bachelorarbeit, eine Masterarbeit oder eine Dissertation, zu unterstützen.
Für die Erstellung habe ich im Wesentlichen genutzt: 
%
\begin{myitemize}[nosep]
\item
M. Kohm, \emph{KOMA-Script}, \cite{kohm:2020}   

\item
H. Voß, \emph{Erstellung einer wissenschaftlichen Arbeit mit \LaTeX{}}, \cite{voss:2021}.

\item
Als Alternative mit den wesentlichen \LaTeX\ Informationen: \textcite{schlosser:2016}.

\end{myitemize}
%
Alles, was ich in den Vorlagen verwende, ist in diesem ReadMe ausführlich beschrieben.
Eine Zusammenfassung zu KOMA-Script und dessen Möglichkeiten gibt es auch noch \href{https://ctan.org/pkg/latex-refsheet}{das \LaTeX\ Reference Sheet} -- bitte nutzen.
Unabhängig davon empfehle ich die Kurzeinführung \textcite{lshort-german} in \LaTeX{}, auch wenn sich diese Anleitung auf eine ältere \LaTeX{}-Version stützt und manches heute nicht mehr erforderlich ist.

Was das Schreiben eines mathematischen Textes betrifft, so ist
%%
\begin{myitemize}[nosep]
\item
\href{https://sites.math.washington.edu/~lind/Resources/Halmos.pdf}{How to Write Mathematics} von P. Halmos 
%%
\end{myitemize}
%
Pflichtlektüre für jeden Mathematiker.
Als Ergänzung dazu bitte auch den \href{https://www.ams.org/notices/200709/tx070901136p.pdf}{Nachruf auf P. Halmos} lesen.
Auf YouTube findet sich auch ein \href{https://www.youtube.com/watch?v=Cy_1JgYfKmE}{Video} dazu.
Dies ist ein Vortrag im Rahmen einer Vorlesungsreihe von D. Knuth zum Thema \enquote{Mathematical Writing} -- es lohnt sich, dieses anzusehen.
Und wer auch noch wissen will, warum es \TeX{} und damit \LaTeX{} gibt: siehe \textcite{knuth:digital}, \textcite{beeton:math} und \textcite{lamport:dmv}.

Des weiteren gehe ich davon aus, dass jeder ein aktuelles \TeX{}-System auf seinen Rechner hat. 
Dieses findet man für Windows oder Linux unter \href{https://tug.org/texlive/}{https://tug.org/texlive/}.
Für Mac OS X gibt es unter \href{https://tug.org/mactex/}{https://tug.org/mactex/} das aktuelle System mit Editor und einem Verwaltungsprogramm für die Literatur.
Eine Alternative ist das System \href{https://www.overleaf.com}{Overleaf}, auf das man mittels eines Browsers online zugreifen kann.
Die entsprechenden Anleitungen und weitergehenden Informationen finden sich auf den angegeben Webseiten.

Die Vorlagen selbst finden sich auf GitHub 
\href{https://github.com/ugroh/AGFA-Master}{https://github.com/ugroh/AGFA-Master} und kann als ZIP-File herunter geladen werden; siehe hierzu das \texttt{ReadMe.md} File auf Github. 

\mytodo{Das ReadMe für AGFA-Sem und dieses hier zusammenfassen}






		% Zusammenfassung
\cleardoubleoddpage 

%%\end{comment}

%% %%%%%%%%%%%%%%%%%%%%%%%%%
%% Mainmatter
%% %%%%%%%%%%%%%%%%%%%%%%%%

\pagenumbering{arabic}
\setcounter{page}{1}
\setcounter{section}{0}
\thispagestyle{empty}

%% -- Die einzelnen Abschnitte
%% -- Über \includeonly steuern, was aktuell bearbeitet wird
%% --
\include{./content/Abschnitt0}	

% !TEX root = ../agfa-readme.tex
%% %%%%%%%%%%%%%%%%%%%%%%%%%%%%%%%%%%%%
%% Section1 in ../agfa-readme.tex
%% Stand: 2022/02/29
%% %%%%%%%%%%%%%%%%%%%%%%%%%%%%%%%%%%%
\thispagestyle{empty}
\dictum[Uwe Seeler]{Ich bin dafür, jetzt mit der Relation erstmal im Dorf zu bleiben.} 
%% %%%%%%%%%%%%%%%%%%%%%%%%%%%%%%%%%%%%
\section{Die Vorlage}\label{sec:section1}
%%
\subsection{Der Aufbau}\label{subsec:aufbau}
Die Vorlage wird als zip-File zur Verfügung gestellt, wobei diese Datei den folgenden Aufbau hat:
%
\begin{description}
\item[./] Im Hauptverzeichnis befindet sich die Datei \lpkg{AGFA-Master.tex}, die als Basis für eine eigene Datei genommen wird.

\item[./content] In diesem Unterverzeichnis stellt man die eigenen Dateien, die den Text enthalten, ein.
Die momentan enthaltenen sollen als Beispiel dienen.

\item[./preamble] Enthält alle Steuerungspakete, die benutzt werden. 
Auf diese wird mittels \lpkg{./preamble/agfa-art.sty} zugegriffen und die im Weiteren beschreiben wird.%
\footnote{In dieser Dokumentation ist auf den Präfix \texttt{./preamble} verzichtet worden.} 

\item[./bib] Enthält eine Musterdatenbank mit einigen Literaturverweisen, die hier genutzt werden.
Der Aufbau, die Pflege und die Nutzung ist von mir in dem \LaTeX\-Tipps~6 beschrieben (siehe \cite{latextipps6})

\item[./texmf]
Bildet die Struktur von \texttt{texmf} ab, das auf einen PC bei einer richtigen \TeX-Installation vorhanden ist: 
Bei Mac OS X findet es sich unter \texttt{~/Library} für Windows \bzw Linux direkt unter dem Home-Verzeichnis.
Ich empfehle dieses zu nutzen, da man dann stets auf alle Steuerungsdateien und die Literaturdatenbank zugreifen kann.  

\end{description}
%
Zur Installation: Wer Overleaf nutzt, der kann das ZIP-File hochladen und das Overleaf-System installiert dieses mit den entsprechenden Unterverzeichnissen.
Das Master-File und seine eigenen Dateien entsprechend umbenennen

Wer die Vorlage lokal auf seinem PC nutzen will, kann natürlich das ZIP-File auf dem PC installieren (in einem geeigneten Unterverzeichnis) und kann danach damit ohne Probleme arbeiten -- ein Nachteil ist es aber, dass man an diese Struktur gebunden ist.

Eleganter ist es, die Pakete im Unterverzeichnisse 
%%
\begin{center}
\texttt{./texmf/tex/agfa}
\end{center}
%%
in das entsprechenden Unterverzeichnisses des eigenen \texttt{texmf}-Unterverzeichnis kopieren, wobei hierfür noch das Unterverzeichnis \texttt{texmf/tex/latex/agfa} angelegt werden muss.
Dabei ist bereits berücksichtigt, dass dann im eigenen System dann der Aufruf über \texttt{texmf} erfolgt, \dh \texttt{./preamble} als Präfix entfällt.
Alles zu \texttt{texmf} erfährt \href{https://tex.stackexchange.com/questions/420620/what-is-texmf-and-what-is-its-relation-to-tex}{man unter diesem Link}.
%%
\subsection{Enthaltene Steuerungsdateien}\label{subsec:dateieninzipfile}
%
Im Einzelnen sind in dem ZIP-File unter \texttt{./preamble/..} folgende Dateien enthalten:

\begin{description}
\item[\cs{agfa-art.sty}]
Über dieses Datei wird (bis auf das Literaturverzeichnis) alles weitere gesteuert und auf die hier weiter aufgeführten Dateien verwiesen.
Diese Datei dient zur Vereinfachung und zur Übersichtlichkeit der Präambel bei.

\item[\cs{agfa-babel.sty}]
Mit Hilfe dieser Datei wird die Unterstützung von Deutsch \bzw Englisch gesteuert; siehe hierzu \vref{subsec:babel}.

\item[\cs{agfa-layout.sty}]
Enthält das Layout des Dokuments, also Seitenüberschriften, Formatierung der Absätze \etc; bitte so belassen; siehe hierzu \vref{subsec:agfa-layout}.

\item[\cs{agfa-hyperef.sty}]
Für Links und Querverweise erforderlich.

\item[\cs{agfa-listen.sty}]
Alles, was für Aufzählungen erforderlich ist.

\item[\cs{agfa-mathtools}]
Mathematische Unterstützung; 


\item[\cs{agfa-defn.sty}]
Abkürzungen,  die die Eingabe des Textes unterstützen.

\item[\cs{agfa-pakete.sty}]
Einige sinnvolle Pakete.

\item[\cs{agfa-theorem.sty}]
Die Theoremumgebungen.

\item[\cs{agfa-biblio.sty}]
Für die Ausgabe des Literaturverzeichnisses.

\end{description} 
%
\subsection{Die Prämbel}\label{subsec:preamble}
%
In der Präambel sind alle Dateien enthalten, die zur \enquote{Steuerung} von \LaTeX{} erforderlich sind.
Diese Dateien enthalten Vorgaben für das Layout, die Schrift, die Sprachunterstützung \etc und sind im Folgenden beschrieben.

Als Dokumentenklasse wurde KOMA-Script genutzt und wir starten daher mit
%
\begin{tcblisting}{title= KOMA-Klasse, listing only}
\documentclass[%	 -- siehe KOMA-Script 
	,toc		= bib 		 
	,parskip		= half-		 
	,headings	= normal			 	 		 
	,numbers		= noenddot		
	,leqno
	,version 	= last
%	,DIV 		= calc
	,titlepage	= true
	,ngerman				% Deutscher Text 
%	,english				% Englischer Text	
	]{scrartcl}			% KOMA Artikelmodus			 
\end{tcblisting}
%
Mittels der Eingabe von \texttt{ngerman} \bzw \texttt{english} steuert man die globale Sprache des Dokuments. 
Lokal kann man dann mittels des Pakets \lpkg{babel} noch andere Sprachen einbinden (siehe \vref{subsec:babel}).
%%
\begin{tcblisting}{title= Das Paket \lpkg{agfa-art}, listing only}
\usepackage[%
	,thmframed			% gerahmte Umgebungen
	,numeric				% Nummeriertes LV
	,urldoi				% URL bzw. DOI unter dem Titel der Referenz
	,lmodern				% lmodern oder
%	,libertinus			% libertinus
	]{./preamble/agfa-art} 
\end{tcblisting}
%%
Über das Paket \lpkg{./preamble/agfa-art-sty} wird alles weitere \enquote{gesteuert} und man gibt seine \enquote{Wünsche} wie immer an.


\begin{myitemize}

\item
Angaben zu einem eventuellen alternativen Schriftsatz (siehe \cref{agfa-font}), obwohl der eingestellte alles erfüllt.

\item
Die Art der Nummerierung in dem Literaturverzeichnis, wobei man bei dem eingestellten Wert bleiben sollte, \texttt{numeric}.

\item
Angaben zum Setzen der Theoremumgebungen, \dh mit oder ohne Rahmen (\texttt{thmframed}).

\item
Unterlegung der Literaturtitel mit den entsprechend Links zu Onlineversionen durch Eingabe von \texttt{urldoi}: (siehe hierzu \vref{subsec:agfa-biblio})
\end{myitemize}
%
Also momentan ist ein deutscher Text, gerahmte Theoremumgebungen und die Hinterlegung von URL's oder DOI's hinter dem Titel des Literaturzitats bedeuten (siehe etwa das Literaturverzeichnis dieses AGFA-ReadMe).

Änderungen und andere Optionen, die KOMA-Script betreffen, können über 
%
\begin{center}
\cs{KOMAoptions\marg{Optionen}} 
\end{center}
%
eingebaut werden; siehe hierzu die entsprechenden Abschnitte in \textcite{kohm:2020}.
Dazu gehört unbedingt die Berücksichtigung einer \enquote{Bindekorrektur} berücksichtigt, indem man bei 
%
\begin{tcblisting}{listing only}
%% -- Für den Druck %% entfernen
% \KOMAoptions{BCOR = 12mm}
\end{tcblisting}
%
das \texttt{\%} entfernt.

%%
\subsection{Die Hauptdatei \texttt{agfa-art.sty}}\label{subsec:agfa-art}
%
Über die Datei \lpkg{agfa-art.sty} werden alle Formatierungsschritte gesteuert.
Ich habe bewusst dies so gemacht, damit die Präambel übersichtlich bleibt. 
Die Eingabe erfolgt über
%
\begin{tcolorbox}
\cs{usepackage}\brackets{agfa-art}
\end{tcolorbox}
%%
%\begin{center}
%\lcmd{usepackage[auswahl]\{agfa-art\} }
%\end{center}
%%
%wobei \oarg{auswahl} sein kann
%
%\begin{enumerate}[(i)]
%
%
%\item
%\emph{thmframed}: Dann werden die mathematischen Umgebungen mit einen Rahmen versehen (siehe etwa Zitat).
%
%\item
%\emph{libertinus} \bzw \emph{lmodern}: Alternative Schriften; siehe \vref{agfa-font}.
%
%\end{enumerate}
Für \LaTeX-Fachleute: Man kann aus dieser Datei natürlich auch eine eigene Klasse machen, habe aber bewusst darauf verzichtet, da das System dann aus meiner Sicht komplexer geworden wäre. 
Wer mehr dazu wissen will: \textcite{goossen:2005} ist eine gute Quelle.
%
\subsection{Sprachunterstützung \texttt{agfa-babel.sty}}\label{subsec:babel}
%
Die Vorlage ist so eingerichtet, dass man sowohl Deutsch als auch Englisch ausgewählt werden kann: Wenn Deutsch, dann bitte in \emph{english} auskommentieren (mit einem \% versehen); falls es eine englische Variante werden soll, dann umgekehrt.
Dies wird dann an alle Pakete, die verwendet werden, weitergereicht und entsprechend genutzt.

In dem \og Paket sind die Pakete \lpkg{babel} und \lpkg{csquotes} enthalten, die die Sprachunterstützung unterstützen, \inkl des richtigen \enquote{Trennungsmuster} für Deutsch:
%
\begin{tcblisting}{listing only}
\usepackage}[english,main=ngerman]{babel}
\babelprovide[hyphenrules=ngerman-x-latest]{ngerman}
\end{tcblisting}
%
Wird \emph{english} ausgewählt, so wird 
%
\begin{tcblisting}{listing only}
\{usepackage}[ngermen,main=english]{babel}
\end{tcblisting}
%
aufgerufen.

Mit Hilfe von des Pakets \lpkg{csquotes} bekommt man nun die richtigen Anführungszeichen für die jeweilige Sprache, die man gewählt hat, also etwa 
%
\begin{tcolorbox}
\cs{usepackage}[autostyle,german=guillemets]\{csquotes\}
\end{tcolorbox}
%
werden in beiden Fällen die richtigen Anführungszeichen gesetzt, also etwa
%
\begin{tcblisting}{title= \enquote{Anführungszeichen Deutsch}}
Richtig: \enquote{Gänsefüßchen}
Und noch richtiger: \enquote{Gänsefüßchen und nochmals \enquote{Gänsefüßchen} im Text}
\end{tcblisting}
%
In beiden Fällen kann man in eine andere Sprache umschalten, etwa von deutsch auf englisch:
%
\begin{otherlanguage}{english}
\begin{tcblisting}{title= \enquote{Anführungszeichen Englisch}}
Now we get \enquote{the right one.}
Additionally: \enquote{Gänsefüßchen and once more \enquote{Gänsefüßchen} in the text.}
\end{tcblisting}
\end{otherlanguage}
%
Wer aber weitere Sprachen nutzen will, muss dieses entsprechend ergänzen.
Details hierzu und wie man umschaltet findet man im Manual zum Paket \lpkg{babel} unter 
\href{https://ctan.ebinger.cc/tex-archive/macros/latex/required/babel/base/babel.pdf}{babel.pdf} oder schaut in \textcite[3.7.2]{voss:2012a} rein.

Falls die Trennung nicht korrekt ist, so hilft \lcmd{hyphenation} in der Präambel, mal in das Manual zu \lpkg{babel} reinsehen (oder in die angegeben Literatur).
%%
\subsection{Eingabe von Abkürzungen: \texttt{agfa-abkuerz.sty}}\label{agfa-abkuerz}
In dieser Datei sind einige Abkürzungen definiert. 
Motivation: Aber was ist mit d.h.? Diese Eingabe ist so falsch, da nach dem Komma ein kleiner Abstand sein soll, also \dh, was man mittels \verb|d.\,h.| eingeben muss.
Da \TeX\ den Punkt nicht als Satzende interpretiert, muss dem System noch mitgeteilt werden, dass der Punkt kein Satzende ist.
Dies gilt auch für u.\,a., z.\,B. \etc

Dies bekommt man mittels des Paketes \lpkg{xspace} und den entsprechenden Definitionen einfach implementiert.%
\footnote{Bitte im Manual \href{https://mirror.informatik.hs-fulda.de/tex-archive/macros/latex/required/tools/xspace.pdf}{xspace.pdf} nachlesen, was es mit dem \cs{xspace} auf sich hat} 
%
\begin{tcblisting}{listing only}
\renewcommand{\dh}{d.\,h.\xspace}
\newcommand{\ua}{u.\,a.\xspace}
\newcommand{\zB}{z.\,B.\xspace}
\newcommand{\og}{o.\,g.\xspace}
\newcommand{\etc}{etc.\xspace}
\newcommand{\bzw}{bzw.\xspace}
\newcommand{\inkl}{inkl.\xspace}
\newcommand{\ia}{i.\,A.\xspace}
\end{tcblisting}
%
etwa ergibt \cs{ua} dann \ua

Für Englisch ist definiert:
%
\begin{tcblisting}{listing only}
\renewcommand{\eg}{e.g.\xspace}
\newcommand{\ie}{i.e.\xspace}
\end{tcblisting}
%
Etwa \cs{eg} ergibt \eg 

Für weitere englische Abkürzungen muss man es entsprechend ergänzen.

Bitte auch beachten: 
\begin{tcblisting}{title= Gedankenstrich und Minuszeichen}
Es ist ein Unterschied, ob ich $ 2 - 1 $ (Minuszeichen) oder - oder -- eingebe.
\end{tcblisting}
%%
Die Eingabe von \enquote{-}, etwa bei \LaTeX{}-Vorlage richtig und die Eingabe von \enquote{--}, etwa im Sinne von  \enquote{von--bis}.
In \TeX{} macht man dieses mit \verb+-+ \bzw \verb+--+; weiteres dazu findet man unter \href{https://typefacts.com/artikel/binde-und-gedankenstrich}{Binde- und Gedankenstrich}.%
\footnote{Nützlich dazu ist auch der \href{https://de.wikipedia.org/wiki/Duden}{Duden}.}
%
\subsection{Das Layout: \texttt{agfa-layout}}\label{subsec:agfa-layout}
\subsubsection{}\label{subsubsec:test}
Die Definitionen zum Layout findet sich in \texttt{agfa-layout.sty} und sollten so belassen werden.

Alle Abschnitte verhalten sich bei 
%
\begin{tcolorbox}
\cs{section}\brackets{Haupttitel}
\end{tcolorbox}
%
und
%
\begin{tcolorbox}
\cs{subsection}\brackets{Untertitel}
\end{tcolorbox}
%
wie gewohnt, nur im TOC gibt es keine Seitenangaben für den jeweiligen Hauptabschnit \cs{section}, da ich dieses für überflüssig halte.
Der einzige Unterschied ist der Befehl
%
\begin{tcolorbox}
\cs{subsubsection}\brackets{}
\end{tcolorbox}
%
Bei diesem wird innerhalb des Hauptabschnittes nur durchnummeriert aber man kann darauf verweisen.
Optional kann man auch einen Titel eingeben, der aber nicht im TOC erscheint.
Dies habe ich aus \textcite{bourbaki:spectrales} übernommen, was ich persönlich gut finde, da es eine weitere Struktur in den Text bringt, diesen aber mit mit weiteren Überschriften überfrachtet.
%
\begin{NoTOCEntry}
%%
\begin{tcblisting}{title= Ein Beispiel}
\section{Die Vorlage}\label{sec:section1}
\subsection{Der Aufbau}\label{subsec:aufbau}
%% --
Die Vorlage wird als zip-File \texttt{AGFA-Master.zip} zur Verfügung gestellt, wobei diese Datei den folgenden Aufbau hat: \ldots
%% --
\subsection{Ein weitere Abschnitt}\label{subsec:weiterer-abschnitt}
\subsubsection{}\label{subsubsec:test}
%% --
Ein weitere Unterabschnitt, der noch einen Unter-Unter-Abschnitt enthält. 
\end{tcblisting}
%%
\end{NoTOCEntry}
%%
\subsubsection{}
Will man auf einen solchen Abschnitt zugreifen, so kann man dieses auch weiterhin mit den üblichen Befehlen machen, d.h. mittels \cs{vref} oder \cs{cref} (siehe hierzu den \LaTeX{}-Tipp 2, \cite{latextipps2}).
Also
%
\begin{tcblisting}{title=Verweise auf \cs{subsubsection} }
siehe etwa \ref{subsubsec:test} in \vref{subsec:weiterer-abschnitt} für weitere Details zu dem Verhalten von \ldots
\end{tcblisting}
%
%%
%% Kopf- und Fußzeile
\subsubsection{}\label{subsec:kopfzeile}
%
Die laufenden Kopfzeilen beinhalten die Überschrift des Hauptabschnittes und die Seitenzahl.
In der Fußzeile befindet sich für den Entwurf, den Namen des Autors und das aktuelle Datum, sodass man dann die verschiedenen Versionen unterscheiden kann.%
\footnote{Wer es komfortabler haben will, den bitte ich GitHub zu nutzen; siehe \href{https://github.com}{GitHub} und den entsprechenden Abschnitt in \textcite{schlosser:2016}}
Bei der Version, die abgegeben wird, muss man  dies in der Präambel auskommentieren.

%
\begin{tcblisting}{listing only}
%% -- für die finale Version die folgenden Zeilen auskommentieren mit % 
\KOMAoptions{footsepline}
\lofoot{Name}
\cofoot{Stand der Arbeit:}
\rofoot{\today}
%% 
\end{tcblisting}
%
%Diese Definition findet man in dem Unterverzeichnis
%
%\setcounter{subsection}{7} % nötig wegen obigen listing-Beispiels
\subsection{Der Schriftsatz: \texttt{agfa-font.sty}}\label{agfa-font}
%
Der Schriftsatz ist eingestellt auf \enquote{Times New Roman}: 
%%
\begin{tcblisting}{title= Die Schrift, listing only}
\usepackage{mathptmx}	% Times New Roman 
	\usepackage[scaled=.90]{helvet}
	\usepackage{courier} 
\end{tcblisting}
%
In dieser Kombination sind alle mathematischen Symbole enthalten und auch passt alles gut zueinander.
Alternativen sind die Schriftsätze \href{https://www.ctan.org/pkg/lm}{\lpkg{lmodern}} \bzw 
\href{https://www.ctan.org/pkg/libertinus}{\lpkg{libertinus}} -- siehe hierzu \vref{subsec:preamble}.
%
%% %%%%%%%%%%%%%%%%%%%%%%%%%%%%%%%%%%%%
\subsection{Ergänzende Pakete: \texttt{agfa-pakete.sty}}\label{subsec:agfa-pakete}
%
Ergänzende Pakete, die sinnvoll sind, werden über die Datei
%
\begin{tcolorbox}
agfa-pakete.sty
\end{tcolorbox}
%
geladen.
Bitte in dieser Datei nachsehen, welche Pakete enthalten sind.
Die zugehörige Dokumentation findet man auf \href{https://www.ctan.org}{https://www.ctan.org} oder man kann  diese mittels des Befehls \cs{texdoc} \meta{Paketname} sich auf dem PC anzeigen lassen.
Nützlich ist hierzu sind auch die Beschreibungen und die Beispiele zu den Paketen in \textcite{voss:2014}.
%
\subsection{Links und Querverweise: \texttt{agfa-hyperef.sty}}\label{subsec:agfa-hyperef}
Dieses Paket ist wie folgt aufgebaut:
%
\begin{tcblisting}{title= agfa-hyperef , listing only}
\RequirePackage{varioref}	
\RequirePackage[breaklinks = true]{hyperref}   
\RequirePackage{cleveref}
\end{tcblisting}
%
Die Pakete \lpkg{varioref} und \lpkg{cleveref} hatte ich bereits vorgestellt und weiteres dazu findet sich in den  \LaTeX{}-Tipps \cite{latextipps2}.

Die Existenz \bzw Farbe der Links wird über
%
\begin{tcblisting}{listing only}
\hypersetup{
	,colorlinks	= true    	%  Für PDF true                                                           
	,urlcolor	= blue      	%  Farbe                                                            
	,citecolor	= blue      	%                                                         
	,linkcolor	= blue		%  
	%  	,hidelinks 			% Vor dem Druck 
							% aktivieren
	}
\end{tcblisting}
%
gesteuert.
Bitte vor dem Druck \texttt{hidelinks} aktivieren.
%
\subsection{Literaturverzeichnis: \texttt{agfa-biblio.sty}}\label{subsec:agfa-biblio}
%
Generelles zur Verwaltung und der Ausgabe der verwendeten Literatur finden sich in den \LaTeX-Tipps \cite{latextipps6}.
Hier ist dieses in der Datei
%
\begin{tcolorbox}
agfa-biblio.sty
\end{tcolorbox}
%
%enthalten, wobei der Zitierstil über \emph{numeric} gesteuert wird.		 
%%
%\begin{tcblisting}{listing only}
%\usepackage[numeric]{./preamble/agfa-biblio}} 
%\end{tcblisting}
%%
enthalten, wobei der Zitierstil \texttt{numeric} ist (bitte beibehalten).%
%\footnote{Wenn man \texttt{numeric} entfernt bekommt man eine Darstellung wie etwa in \textcite{schaefer:bl}, was aber aus meiner Sicht nur für große Dokumente sinnvoll ist}

Der Aufruf der Literaturdatenbank erfolgt über
%
\begin{tcblisting}{listing only}
\addbibresource{agfa-bib.bib}	
\end{tcblisting}{listing only}
%
(die Endung \texttt{bib} nicht vergessen) und über
%
\begin{tcblisting}{listing only}
\ExecuteBibliographyOptions}{%
	,backref		= true		% sinnvoll
 	,url		= true		% falls vorhanden
 	,doi		= false		% dito
	,eprint		= false		% dito
	}	
\end{tcblisting}
%	
kann man noch das Verhalten steuern.
Dabei halte ich insbesondere \texttt{backref} bei der Erstellung der Arbeit für wichtig, da nur solche Literatur in das Verzeichnis aufgenommen werden soll, das man auch tatsächlich verwendet hat.

Noch eine Anmerkung: Will man die zu den zitierten Arbeiten eventuelle vorhandenen URL oder DOI Eintragungen bei den Titel hinterlegen, so kann man die Option \texttt{urldoi} angeben. 
%%
%\begin{tcblisting}{listing only}
%\usepackage[numeric,urldoi]{./preamble/agfa-biblio}} 
%\end{tcblisting}
%%
Dann kann man die Titel anklicken und wird direkt zu der entsprechenden Seite geführt (siehe das Literaturverzeichnis)
Sicherlich sinnvoll bei der PDF-Variante des Dokuments, für den Druck macht es keinen Sinn. 

\subsection{Eigene Dateien: Das Unterverzeichnis \texttt{./content}}\label{subsec:content}
%
Die eigene Dateien finden sich in dem Unterverzeichnis
%
\begin{tcolorbox}
./content
\end{tcolorbox}
%
Im Folgenden muss daher der jeweilige Name stets um diesen Präfix ergänzt werden.

Die Titelseite ist unter
%
\begin{tcolorbox}
0-AGFA-titel.tex
\end{tcolorbox}
%
zu finden und ist so gestaltet, dass alle relevanten Daten enthalten sind; einfach dort entsprechend ergänzen.

Des weiteren empfehle ich die Hauptabschnitte in separate Dateien auszulagern, wie ich dieses in dem Muster gemacht habe und dies via \cs{include} einzubinden, wobei man mittels \cs{includeonly} steuern kann was aktuell bearbeitet wird.
Dieser \cs{include}-Befehl erzeugt zwar stets eine neue Seite, aber dies finde ich auch besser, selbst bei kleineren Arbeiten.
Dies habe ich so auch in dieser Zusammenstellung gemacht, wobei bei mir die einzelnen Dateien wie folgt anfangen:
%
\begin{tcblisting}{title= Beispiel, listing only}
% !TEX root = ../agfa-readme.tex
%% %%%%%%%%%%%%%%%%%%%%%%%%%%%%%%%%%%%%
%% Section1 in ../agfa-readme.tex
%% Stand: 2022/02/29
%% %%%%%%%%%%%%%%%%%%%%%%%%%%%%%%%%%%%
\thispagestyle{empty}
\dictum[Uwe Seeler]{Ich bin dafür, jetzt mit der Relation erstmal im Dorf zu bleiben.} 
%% %%%%%%%%%%%%%%%%%%%%%%%%%%%%%%%%%%%%
\section{Die Vorlage}\label{sec:section1}
\end{tcblisting}
%%

%
\begin{tcblisting}{title= \cs{include} und \cs{includeonly},listing only}
%% --- Include
\includeonly{%
	./content/AGFA-Section-1	,	
%	./content/AGFA-Section-1 	
	}
...
\begin{document}
....
% wird aufgerufen
\include{./content/AGFA-Section-1}	
% wird ignoriert
\include{./content/AGFA-Section-2}	
...
\end{tcblisting}
%
In diesem Fall wird nur die Datei \emph{AGFA-Section-1.tex} eingebunden.%
\footnote{Oder eigene eindeutige Namen vergeben}
Hat man aber vorher eine Umwandlung mit allen Dateien gemacht, bleiben Querverweise und Nummerierungen erhalten, auch wenn man die Abschnitte separat aufruft.
%%
				% Erster Abschnitt
% !TEX root = ../agfa-readme.tex
%% %%%%%%%%%%%%%%%%%%%%%%%%%%
%% Section2 in ../agfa-readme.tex
%% Stand: 2023/01/29
%% %%%%%%%%%%%%%%%%%%%%%%%%%%
\dictum[Erich Kästner]{Denkt an das fünfte Gebot: Schlagt eure Zeit nicht tot!}
\thispagestyle{empty}
\section{Mathematik und mehr}\label{sec:section2}
\subsection{Aufzählungen: Das Paket \texttt{agfa-listen.sty}}\label{subsec:agfa-listen}
%
Mit Hilfe des Paketes \lpkg{enumitem} ist es einfach \emph{Aufzählungen} zu erstellen, die wir in 
%
\begin{tcolorbox}
agfa-listen.sty
\end{tcolorbox}
%
zusammengefasst und erweitert haben.
Dazu einige Beispiele: 
%
\begin{tcblisting}{title= Nummerierte Listen, sidebyside }
\begin{enumerate}[(i)]
\item
Erstes Item
\item
Zweites Item
\item
\ldots
\end{enumerate}
\end{tcblisting}
%
\begin{tcblisting}{title= Äquivalenzen, sidebyside }
\begin{enumerate}[(a)]
\item
Erstes Item
\item
Zweites Item
\item
\ldots
\end{enumerate}
\end{tcblisting}
%
\begin{tcblisting}{title= Inline}
\begin{enumerate*}[(1)]
\item
Dies ist ein Typoblindtext. 

\item
Oder manchmal Sätze, die alle Buchstaben des Alphabets enthalten - man nennt diese Sätze \enquote{Pangrams}. 

\end{enumerate*}
\end{tcblisting}
%
Die Erweiterungen finden sich in dem \og \TeX{}-File, etwa
%
\begin{docEnvironment}{myequivalent}{\oarg{options}}
\end{docEnvironment}
%
Dies ist die Umgebung für äquivalente Aussagen in Theorem, Sätzen etc., wobei die Eingabe wie bei Listen üblich mit \lcmd{item} erfolgt.
Dies eignen sich für Aufzählungen, bei denen die einzelnen Items länger sind und über mehrere Zeilen gehen.
Mittels der Option \texttt{nosep} kann man etwa steuern, ob die Items kompakt gesetzt werden sollen (kann man immer nutzen, auch bei den ersten Beispielen).

%
\begin{dispExample*}{title=Äquivalenz}
\begin{myequivalent}
	\item
	Erstes Item. 
	\item
	Zweites Item.
  \begin{myequivalent}[nosep]
		\item
		Subitem; enger Abstand
		\item
		Subitem 
  \end{myequivalent}
	\item
	Drittes Item.
\end{myequivalent}
\end{dispExample*}
%
oder für nummerierte Listen entsprechend
%
\begin{docEnvironment}
	{myenumerate}{\oarg{options}}
\end{docEnvironment}
%
Beispiel mit \texttt{myequivalent}:%
\footnote{Solche \enquote{Textfüller} findet man mit Hilfe von \href{https://www.blindtextgenerator.de}{https://www.blindtextgenerator.de} oder den diversen Paketen, etwa \lpkg{blindtext}.}
%
\begin{myequivalent}
	\item
	Jemand musste Josef K. verleumdet haben, denn ohne dass er etwas Böses getan hätte, wurde er eines Morgens verhaftet. »Wie ein Hund!« sagte er, es war, als sollte die Scham ihn überleben. Als Gregor Samsa eines Morgens aus unruhigen Träumen erwachte, fand er sich in seinem Bett zu einem ungeheueren Ungeziefer verwandelt. 
	
	\item
	Und es war ihnen wie eine Bestätigung ihrer neuen Träume und guten Absichten, als am Ziele ihrer Fahrt die Tochter als erste sich erhob und ihren jungen Körper dehnte. »Es ist ein eigentümlicher Apparat«, sagte der Offizier zu dem Forschungsreisenden und überblickte mit einem gewissermaßen bewundernden Blick den ihm doch wohlbekannten Apparat. 
	
	\item
	Sie hätten noch ins Boot springen können, aber der Reisende hob ein schweres, geknotetes Tau vom Boden, drohte ihnen damit und hielt sie dadurch von dem Sprunge ab. In den letzten Jahrzehnten ist das Interesse an Hungerkünstlern sehr zurückgegangen. Aber sie überwanden sich, umdrängten den Käfig und wollten sich gar nicht fortrühren.Jemand musste Josef K. verleumdet haben, denn ohne dass er etwas Böses getan hätte, wurde er eines Morgens verhaftet. »Wie ein Hund!« sagte er, es war, als sollte die Scham ihn überleben. Als Gregor Samsa eines Morgens aus unruhigen Träumen erwachte, fand er sich
\end{myequivalent}
%
im Gegensatz zu \cs{enumerate[(a)]}
%
\begin{enumerate}[(a)]
	\item
	Jemand musste Josef K. verleumdet haben, denn ohne dass er etwas Böses getan hätte, wurde er eines Morgens verhaftet. »Wie ein Hund!« sagte er, es war, als sollte die Scham ihn überleben. Als Gregor Samsa eines Morgens aus unruhigen Träumen erwachte, fand er sich in seinem Bett zu einem ungeheueren Ungeziefer verwandelt. 
	
	\item
	Und es war ihnen wie eine Bestätigung ihrer neuen Träume und guten Absichten, als am Ziele ihrer Fahrt die Tochter als erste sich erhob und ihren jungen Körper dehnte. 
	»Es ist ein eigentümlicher Apparat«, sagte der Offizier zu dem Forschungsreisenden und überblickte mit einem gewissermaßen bewundernden Blick den ihm doch wohlbekannten Apparat. 
	
	\item
	Sie hätten noch ins Boot springen können, aber der Reisende hob ein schweres, geknotetes Tau vom Boden, drohte ihnen damit und hielt sie dadurch von dem Sprunge ab. 
	In den letzten Jahrzehnten ist das Interesse an Hungerkünstlern sehr zurückgegangen. 
	Aber sie überwanden sich, umdrängten den Käfig und wollten sich gar nicht fortrühren.
	Jemand musste Josef K. verleumdet haben, denn ohne dass er etwas Böses getan hätte, wurde er eines Morgens verhaftet. »Wie ein Hund!« sagte er, es war, als sollte die Scham ihn überleben. 
	Als Gregor Samsa eines Morgens aus unruhigen Träumen erwachte, fand er sich \ldots
	
\end{enumerate}
%
\subsection{Das Paket \texttt{agfa-theorem}}\label{subsec:agfa-theorem}
%%
\begin{center}
\begin{table}
\caption{Die Theoremumgebungen}\label{table:thm-umgebung}
\begin{tabular}{lcll}\toprule
THM-Umgebung	&   & Ersetzung dtsch. & Ersetzung engl. \\ \hline 
\multicolumn{3}{c}{Mit Rahmen:}\\ 
theorem oder thm 	&   & Theorem \\
proposition	oder prop	&   & Satz & Proposition \\ 
lemma	&   & Lemma \\ 
corollary oder cor	&   & Korollar & Corollary \\ 
\multicolumn{3}{c}{{Immer ohne Rahmen: } }\\ 
definition	oder defn &   & Definition \\
remark oder rem &   & Anmerkung & Remark \\
proof	&   & Beweis & Proof \\
\bottomrule
\end{tabular}
\end{table}
\end{center}

%
\begin{docEnvironment}{THM-Umgebung}{}
\end{docEnvironment}
%

In diesem Paket finden sich die Umgebungen für Theoreme, Lemmata, Korollare \etc und man kann mittels der Option \texttt{thmframed} wählen, ob man einen Teil dieser Umgebungen eingerahmt haben will, was manchmal etwas Auflockerung in die mathematische Darstellung bringt.
Wählt man Englisch als Sprache, so wird dies entsprechend berücksichtigt.
Das Setzen der Umgebungen ist also immer gleich, man muss dem System nur sagen, was man will, \dh man ersetzt \texttt{THM-Umgebung} durch die entsprechende Definition (siehe \vref{table:thm-umgebung}). 

\begin{tcblisting}{title= Umgebung für Theoreme}
%
\begin{theorem}\label{thm:theorem}
Stets ist $ \int_{ 0 }^{ 1 } f(s) \ds \neq 0 $ für eine stetige positive Funktion $ \neq 0 $.
\end{theorem}
%
\begin{corollary}\label{cor:folgerung}
Stets ist $ f \mapsto \int_{ 0 }^{ 1 } \abs{ f(s) }\ds $ eine Norm auf dem Vektorraum 
$ C\interval{0,1} $.
\end{corollary}
%
\end{tcblisting}
%
und mittels den üblichen Befehlen (siehe das entsprechende \LaTeX{}-Tipps dazu) kann man darauf verweisen:
%
\begin{tcblisting}{title=Querverweis} 
Wir verweisen auf \vref{thm:theorem} ...
\end{tcblisting}
%
%%
%Für eine englische Variante also:
%%
%\begin{otherlanguage}{english}
%\begin{tcblisting}{title = Englische Umgebungen} 
%\begin{proposition}\label{prop:proposition}
%\ldots  $ \int_{ 0 }^{ 1 } f(s) \ds \neq 0 $ \ldots $ \neq 0 $.
%\end{proposition}
%%
%\begin{corollary}\label{cor:corollary}
%\ldots $ f \mapsto \int_{ 0 }^{ 1 } \abs{ f(s) }\ds $ defines a norm on  
%$ C\interval{0,1} $.
%\end{corollary}
%\end{tcblisting}
%\end{otherlanguage}
%
\subsection{Einige Abkürzungen: \texttt{agfa-defn}}\label{agfa-defn}
In dem Paket \texttt{agfa-defn} habe ich einige Abkürzungen eingestellt, die aus meiner Sicht nützlich sind und die Eingabe von \TeX\ erleichtert.
Diese habe ich hier nicht weiter im Detail aufgeführt, aber ein Blick in dieses Datei ist sicherlich nützlich.

Vorab noch eine Anmerkung zur Eingabe eines mathematischen Textes: Auch hierfür gelten einige typographische Regeln, die zu beachten sind.
Eine Kurzfassung findet man etwa in \textcite{nadler:formelsatz}, in \textcite[Kap. 9.1]{voss:2012a} und ausführlicher, versehen mit vielen Beispielen in \href{https://archiv.dante.de/DTK/PDF/komoedie_1997_1.pdf}{Marion Neubauer: \emph{Feinheiten bei wissenschaftlichen PublikationenPublikationen}}.%
\footnote{Der Link ist hinterlegt und der Artikel findet sich ab Seite 25 und der zugehörige erste Teil \href{https://archiv.dante.de/DTK/PDF/komoedie_1996_4.pdf}{findet sich hier}}

In \vref{tab:definitionen} finden sich einig Beispiele dazu.
Den Rest bitte in der Datei \texttt{agfa-defn.sty} nachsehen.
Falls mal ein mittels \cs{newcommand} definierter eigener Befehl nicht klappt (\ldots bereits definiert), dann unbedingt reinsehen.

Noch ein Hinweis: Ich habe die \cs{var}-Varianten \enquote{umgetauft}: also \cs{phi} gibt $ \phi $ und \cs{varphi} gibt $ \varphi $. 
Entsprechend auch bei den anderen aufgeführten Zeichensätze, die eine \cs{var}-Variante haben. 
%
\begin{table}
\begin{tabular}{l  >{$}l<{$} l l  >{$}l<{$} l}
\cs{N}	&	\N 	& 		& 	\cs{phi} 			& 		\phi 		&  	\\
\cs{Z}	&   \Z	&	   	& 	\cs{psi}			&      	\psi  			\\
\cs{Q}	&   \Q 	&		&	\cs{epsilon}		&       \epsilon    	\\
\cs{R}	&	\R	&		&	\cs{rho}			&	 	\rho 			\\
\cs{C}	&	\C	&		&	\cs{theta}			&	 	\theta			\\
\cs{P} 	&	\P	& Potenzmenge	& \cs{geq}		&	 	\geq 			\\
\cs{diff}\brackets{\cs{mu}}	& \diff{\mu}  &	& \cs{leq}	&	 	\leq  \\
\cs{dt}	&	\dt &	Eulersche Zahl		& 	\cs{eu} 	& \eu  &    \\
\cs{ds} & 	\ds	& 	Imaginäre Einheit	&	\cs{im}	& \im &  \\
\cs{e}  &  \e &&&&\\ \hline 
\cs{finv}\marg{arg}	& \finv{arg} & Inverse Mengenfunktion & etwa & \finv{g}(A) \\
\cs{Kern}\marg{arg}	& \Kern{arg}  & Kern & etwa & \Kern{T} \\
\cs{Bild}\marg{arg}	& \Bild{arg} & Bild  & etwa & \Bild{T}\\
\cs{Fix}\marg{arg}& \Fix{arg} & Fixraum & etwa & \Fix{T}\\ 
%\cs{Sp}\marg{arg} & \Sp{arg}  & Spektrum & etwa & \Sp{T} \\
\end{tabular} 
\caption{Einige Abkürzungen aus \texttt{agfa-defn}}\label{tab:definitionen}
\end{table} 
%\begin{table}[!htb] 
%\ttabbox[0.5\linewidth]%
%{\begin{subfloatrow}[2]
%	\ttabbox[\FBwidth]{\caption{\cs{var}\ldots}\label{subtab:tab1}}%[\FBwidth]
%	{\begin{tabular}{l  >{$}l<{$} }%\hline 
%	\cs{phi}	\footnote{Wer aber $ \varphi $ unbedingt haben will: \cs{varphi} eingeben (analog für die anderen griechischen Buchstaben).}		&      \phi   		\\
%	\cs{psi}			&      \psi   		\\
%	\cs{epsilon}		&       \epsilon    \\
%	\cs{rho}			&	 	\rho 		\\
%	\cs{theta}		&	 	\theta		\\
%	\cs{geq}			&	 	\geq 		\\
%	\cs{leq}			&	 	\leq 		\\ %\hline
%	\end{tabular}}
%%
%	\ttabbox[\Xhsize]{\caption{Zahlen etc}\label{subtab:tab2}}%[\Xhsize]
%	{\begin{tabular}{l  >{$}l<{$} }%\hline 
%	\cs{N}	&	    \N   \\
%	\cs{Z}	&	    \Z   \\
%	\cs{Q}	&	    \Q   \\
%	\cs{R}	&	    \R   \\
%	\cs{C}	&	    \C   \\ % \hline 
%	\cs{1}	&		\1	\footnote{Eins einer Algebra} \\
%	\end{tabular}}
%%
%\end{subfloatrow}}
%{\caption{Erste Definitionen}\label{tab:erste-def}}
%\end{table}
%%
%\begin{table}[!htb] 
%\ttabbox[0.5\linewidth]%
%{\begin{subfloatrow}[2]
%	\ttabbox[\FBwidth]{\caption{Diff, \ldots}\label{subtab:tab3}}%
%	{\begin{tabular}{l  >{$}l<{$} l}
%	\cs{diff}\brackets{\cs{mu}}	 &	    \diff{\mu}  &  \\
%	\cs{dt}, \cs{ds}, \ldots 	 &	    \dt, \ds, \ldots  & \\
%	\cs{e}					& \e  & Eulersche Zahl	   \\
%	\cs{im}					& \im & Imaginäre Einheit \\
%	\cs{P} 					& \P & Potenzmengensymbol \\
%	\end{tabular}}
%%
%	\ttabbox[\Xhsize]{\caption{Sonstiges}\label{subtab:tab4}}
%	{\begin{tabular}{l  >{$}l<{$} l}
%	\cs{finv}\brackets{f}	& \finv{f}(A) & Inverse Mengenfunktion \\
%	\cs{Kern}	& \Kern{T}  & Kern \\
%	\cs{Bild}	& \Bild{T} & Bild  \\
%	\cs{Fix}		& \Fix{T} & Fixraum \\
%	\end{tabular}}
%
%\end{subfloatrow}}
%{\caption{Weitere Definitionen}\label{tab:zweite-def}}
%\end{table}
%
\subsection{Definitionen in \texttt{agfa-mathtools.sty}}\label{agfa-mathtools}
In dieser Datei befinden sich Tools auf Basis das Pakets \lpkg{mathtools}.
%
\minisec{Norm: }
\begin{docCommands}[%
% 	,doc description = Norm
	,doc parameter = \marg{MathSymbol}
	,doc name = norm	] 
	{%
	,doc name = norm*}	
\end{docCommands}
%
Setzt \meta{MathSymbol} $ x $ in Normzeichen: $ \norm{x} $, wobei die Sternvariante die Länge der Norm an die Umgebung anpasst.
%%
\begin{dispExample*}{title= Beispiel für die Norm}
\ldots $ \norm{} $: So bezeichnet etwa  $ \norm{x} $ die Norm von $ x $ und die Sternvariante passt alles in der Größe an: 
%
\[
	\norm*{\frac{ 1 }{ 1 + t^{ 2 } } }
\]
%
\end{dispExample*}

\minisec{Absolutbetrag: }
%
\begin{docCommands}[%
%	,doc description = Absolutbetrag
	,doc parameter = \marg{MathSymbol}
	,doc name = abs
	] {
		,{ }
		,doc name = abs*}	
\end{docCommands}
%
Setzt \meta{MathSymbol} in beidseitige Betragsstriche $ \abs{\text{\meta{MathSymbol}}} $, wobei die Sternvariante die Länge der Betragsstiche anpasst.

%
\begin{dispExample*}{title= Beispiel für den Absolutbetrag }
Text $ \int_{ \R } \abs{f(t)} \dt $ Text \ldots
%
\[
	g(t) = \abs*{\frac{ 1 }{ 1 - t^{ 2 } } } \, , \quad t \in \R \, .
\]
%
\end{dispExample*}
%
\minisec{Intervalle: }

\begin{center}
\begin{tabular}{l  l >{$} l <{$}}
\docAuxCommand{interval\brackets{a,b}} 	& abgeschlossenes Intervall 
					&   \interval{ a,b}    	\\
\docAuxCommand{ointerval\brackets{a,b}}: 	& offenes Intervall 
					&   \ointerval{a,b}   	\\ 
\docAuxCommand{rointerval\brackets{a,b}}: 	& rechts offenes Intervall 
					&   \rointerval{a,b}  	\\ 
\docAuxCommand{lointerval\brackets{a,b}}: &  links offenes Intervall 
					&   \lointerval{a,b}   	
\end{tabular}
\end{center}


\begin{dispExample*}{title= Beispiel für die Intervalle}
\ldots es ist $ \int_{-1}^{1} \abs{f(t)} \dt $ das Integral der Funktion $ f $ auf dem Intervall $ \interval{-1,1} $ \ldots

Aber existiert auch das Integral der Funktion $ g $ mit
%
\[
	g(t) = \abs*{\frac{1}{1 - t^{ 2 }}} \, , \quad t \in \ointerval{-1,1 } \, .
\]
%
\end{dispExample*}
%%
\minisec{Einfachere Eingabe von Klammern \etc}
Will man etwa Klammern \enquote{ ( \ldots ) } der Größe an dem anpassen, was zwischen ihnen steht, muss man gewöhnlich mit \cs{big\ldots} arbeiten. 
Dies kann man sich ersparen, da ein Macro eingearbeitet ist, dass einem diese Arbeit erspart. 
%
\begin{dispExample*}{title=Ein Beispiel}
Also 
%
\[
( \frac{1}{1 - \sum_{j=1}^{n}r_{j}} )
\]
%
oder
%
\[
\left] \frac{1}{1 - \sum_{j=1}^{n}r_{j}} \right]
\]
%
\bzw
%
\[
\lointerval{ \frac{1}{1 - \sum_{j=1}^{n}r_{j}} } 
\]
%
eingeben; wie es aussieht sieht man unten.
\end{dispExample*}
%
\minisec{Weitere Tools: }
Der Ausdruck $ 1/2 $ in einem Fließtext ist nicht schön aber $ \nfrac{1}{2} $ ist es schon. 
Ebenso ist $ \tfrac{E}{F} $ besser als E/F. 
Es ist $ \interior{A} $ ist das Innere einer Menge eines topologischen Raums.
Eine Übersicht ist in der folgenden Tabelle enthalten.
%
\begin{center}
\begin{tabular}{l  l l }
\docAuxCommand{tfrac\brackets{E}\brackets{F}} 	&   &   \tfrac{E}{F}   	\\
\docAuxCommand{nfrac\brackets{a}\brackets{b}} 	&   &   $ \nfrac{a}{b} $	 \\
\docAuxCommand{interior\brackets{A}}  	&   &   $ \interior{A} $	 \\ 	
\end{tabular}
\end{center}

Und immer daran denken: $ \dfrac{ \pi }{ 2 } $ geht nur so  und nicht so $ \pi/2 $.
%%

				% Zweiter Abschnitt

% !TEX root = ../AGFA-ReadMe.tex
%% %%%%%%%%%%%%%%%%%%%%%%%%%%%%%%
%% ./content/Abschnitt3
%% Stand: 2022/03/08
%% %%%%%%%%%%%%%%%%%%%%%%%%%%%%%% 
\thispagestyle{empty}
\section{Testen der Definitionen}
\subsection{Test der Listen}\label{subsec:listen}
%%
\begin{myenumerate}
\item
Aufzählung
\item
Aufzählung
\end{myenumerate}
%%
\begin{myequivalent}
\item
Äquivalent
\item
Äquivalent
\end{myequivalent}
%%
\begin{myitemize}
\item
Punkte
\item
Punkte
\end{myitemize}
%%
\begin{mynumber}
\item
Nummeriert
\item
Nummeriert
\end{mynumber}
%%
\subsection{Test der mathematischen Umgebungen}\label{subsec:mumgebungen}
Schon seit vielen hundert Jahren eines der schönsten Ergebnisse der Mathematik.
%
\begin{theorem}\label{thm:theorem1}
In einem rechtwinkligen Dreiecke mit den Seiten $ a $, $ b $ und der Hypothenuse $ c $ gilt stets

\begin{equation}\label{eq:pythagoras}
	a^{ 2 } + b^{ 2 } = c^{ 2 } \, .
\end{equation}
\end{theorem}
%
\begin{proof}
Für den Beweis verweisen wir auf die Literatur, etwa \textcite{efhn:2016}
\end{proof}
%%
\begin{corollary}\label{cor:folgerung}
Hieraus folgt dann 

	%
\[
	a^{ 2 } + b^{ 2 } = c^{ 2 } \, .
\]
%


\end{corollary}
%
\begin{proposition}\label{prop:prop}
Und nun ein kleiner Satz als Ergänzung 
\end{proposition}
%
\begin{lemma}
Zuvor aber ein Lemma
\end{lemma}
%
\begin{remark}
Eine Anmerkung
\end{remark}
%%
Mal sehen, wie die Eulersche Zahl und die imaginäre Einheit aussehen.
%%
\begin{theorem}\label{thm:eim}
\begin{equation}\label{eq:pythagoras}
\eu^{ 2 \pi \im } = -1 
\end{equation}
\end{theorem}
%%
Nun ein Seitenvorschub \ldots
%
\newpage
%%
\subsection{Querverweise}\label{subsec:referenzen}

\begin{dispExample*}{title=Querverweise}
\myquestion{Funktionieren alle Querverweise?}
\myanswer{Ja, sie funktionieren}
\mytodo{Noch viel zu tun}
\end{dispExample*}
%
Zunächst auf die Eulersche Zahl \vref{thm:eim} und dann auf Gleichung~\eqref{eq:pythagoras} in \vref{thm:theorem1}.
%
\subsection{Sinnvolle Literatur zu \LaTeX}
%
\begin{verbatim}
Mal ansehen: \textcite{l2tabu} und \textcite{lshort-german} 
\bzw \textcite{latex-refsheet} für all die Befehle 
und Möglichkeiten.
Wie man sieht, sind die Links auf die Dokumente hinterlegt.	
\end{verbatim}
%%
Mal ansehen: \textcite{l2tabu} und \textcite{lshort-german} \bzw \textcite{latex-refsheet} für all die Befehle und Möglichkeiten.
Wie man sieht, sind die Links auf die Dokumente hinterlegt.

Oder auch möglich: %\ldots \cref{thm:theorem1}, \vref{eq:pythagoras} \ldots.
\begin{dispExample*}{title=Querverweise}
\ldots \cref{thm:theorem1}, \vref{eq:pythagoras} \ldots.
\end{dispExample*}
%% --
				% Der Testteil

\cleardoubleoddpage 

%% %%%%%%%%%%%%%%%%%%%%%%%%
%% Backmatter
%% %%%%%%%%%%%%%%%%%%%%%%%%

\thispagestyle{empty}
\pagenumbering{roman}
\setcounter{page}{1}

%% -- Literaturverzeichnis
%%
%\nocite{}		% für Literatur, die nicht zitiert wurde; 
				% wenig sinnvoll dieses zu tun; siehe ReadMe.pdf
				
\RaggedRight		% Kein Blocksatz
\printbibliography

%% --
\end{document}
