% !TEX root = ../SEM-ReadMe.tex
%% -----------------------------
%% Stand: 2023/01/27
%% -----------------------------
\section{Die Muster \TeX{}-Datei}
\subsection{Die Vorlagen}
Ich habe vier Dateien erstellt:

\begin{myitemize}
	
	\item
	\texttt{SEM-Muster.tex}: Diese dient als Vorlage für kleinere Ausarbeitungen, etwa für eine Hausarbeit oder für den Vortrag eines Proseminars oder eines Seminars. 
		Man darf nur diese Datei mit dem \LaTeX{}-Compiler bearbeiten. 
		Die anderen sind sog.\@ \texttt{include}-Dateien, die Makros enthalten.
		
	\item
	\texttt{SEM-Beamer.tex}: Eine Datei, mit der Sie sicherlich Ihren Vortrag mal konzipieren können.
	Einfach mal reinschauen.
	
	\item
	Im Unterverzeichnis \texttt{preamble} befinden sich:
	
	\begin{myenumerate}

	\item
	\texttt{SEM-art.tex}: Beinhaltet das Layout, einige Definitionen für mathematische Umgebungen etc.\@ 
	Details werden in \vref{tab:mathematisches} besprochen.. 
	
	\item
	\texttt{SEM-defn.tex}: Beinhaltet einige Definitionen für Abkürzungen \etc, siehe \vref{tab:generelles}.
	
	\item
	Für eigene Definitionen bitte die Datei \texttt{My-defn.tex} nutzen aber vorher in die beiden \og reinschauen, was wie definiert ist \bzw wird. 
	
	\end{myenumerate}
	
	diese werden via \cs{input}\brackets{name-der-datei} eingebunden.
	
	\item
	Im Unterverzeichnis \texttt{content} befinden sich die Dateien mit dem fachlichen Inhalt, die ebenso über \cs{input}\brackets{name-der-datei} eingebunden werden.
	In unserem Fall \texttt{./ug-Master.tex} unsere Ausführungen und ale weiteres Beispiel \texttt{./MeinText.tex}, in dem man seine eigenen Ausführungen eintragen kann (oder jeder andere Namen für die vorhandene Datei).
		
\end{myitemize}

Man kann den Inhalt von \texttt{./preamble} auch in das  \texttt{texmf}-Verzeichnis unter \texttt{latex} kopieren; dann hat man alles stets zur Verfügung.
Falls der Sinn des \texttt{texmf}-Verzeichnises nicht bekannt ist: Einfach \href{https://www.overleaf.com/learn/latex/Articles/An_introduction_to_Kpathsea_and_how_TeX_engines_search_for_files
\%23Table_listing_Kpathsea_.E2.80.9Cconfig_variables.E2.80.9D}{mal diese Übersicht dazu lesen}.
\begin{remark}
Ein Hinweis für Mac-User: Nach der Installation von \TeX{} gibt es unter \texttt{$\tilde{}$/Library} ein Unterverzeichnis \texttt{texmf/tex/latex}.
Dort gehören die beiden letzten Dateien hin und werden von \LaTeX{} gefunden.

Für Windows-Nutzer: Bitte unbedingt \href{http://texlive.tug.org/texlive/}{texlive} nutzen und \href{http://texlive.tug.org/texlive/windows.html}{die Anleitung lesen}.
\end{remark}
%
\begin{remark}
Für die Nutzer von Overleaf: Bitte über \texttt{Code} rechts oben die zip-File herunterladen und als neues Projekt nach Overleaf kopieren. 
Es wird dann ein Verzeichnis mit dem gleichen Namen angelegt und man kann die Dateien ohne weiteren Installationsaufwand nutzen.
\end{remark}
% ------------------
\subsection{Die Definitionen}\label{subsec:definitionen}
Folgendes ist vordefiniert und findet sich in der \texttt{SEM-defn.tex} Datei.%
\footnote{Die folgenden Tabellen sind mit dem Paket \texttt{longtable} \cite{longtable} gesetzt worden. 
Details hierzu findet man in dem \og Manual oder in \textcite{voss:2012a}.}
% ------------------------------------------------------------------------------
\begin{center}
  \begin{longtable}{@{} lcl @{}}
  \caption{Generelles}\label{tab:generelles} \\
%-
  \toprule
  Die Eingabe & ergibt die  & folgende Ausgabe \\ 
  \toprule
  \endfirsthead  
%-
  \multicolumn{3}{@{}l}{\small\ldots\emph{Fortsetzung}} \\
  \toprule
  Die Eingabe & ergibt die  & folgende Ausgabe \\ 
  \toprule
  \endhead
%-
  \multicolumn{3}{@{}r}{\small{\emph{Fortsetzung nächste Seite}}\ldots} \\
  \endfoot
  \endlastfoot
% ------------------------------------------------------------------------------
    \emph{Anführungszeichen:} \\
    \verb|\enquote{Text}| & $ \to $ & \enquote{Text} \\ 
    \verb|\enquote{\ldots\enquote{Text}\ldots}| & $ \to $ & \enquote{\ldots\enquote{Text}\ldots} \\ 
    \midrule 
    \emph{Abkürzungen:} \\
    \verb|\zB| & $ \to $ & \zB \\ 
    \verb|\dh| & $ \to $ & \dh \\
     \verb|\og| & $ \to $ & \og \\
    \verb|\etc| & $ \to $ & \etc \\
    \verb|\bzw| & $ \to $ & \bzw \\
    \midrule
    \emph{Bindestriche:} \\
    \verb|-| & $ \to $ & Cauchy-Schwarz \\
    \verb|--| & $ \to $ & 1--10 \\
    \verb|$ - x $ | & $ \to $ & $ - x $ \\
%    \verb|$ - x $\) $| & $ \to $ & \( - x \) \\
% ------------------------------------------------------------------------------
    \bottomrule
  \end{longtable}
\end{center}
% 
\begin{remark}
Bei den Abkürzungen muss nach dem ersten Punkt einer kleiner Abstand eingehalten werden (laut Duden).
Dies ist hier eingehalten worden.
\end{remark}

%
\begin{center}
  \begin{longtable}{@{} lcl @{}}
  \caption{Mathematisches}\label{tab:mathematisches} \\
%-
  \toprule
  Die Eingabe & ergibt die  & folgende Ausgabe \\ 
  \toprule
  \endfirsthead  
%-
  \multicolumn{3}{@{}l}{\small\ldots\emph{Fortsetzung}} \\
  \toprule
  Die Eingabe & ergibt die  & folgende Ausgabe \\ 
  \toprule
  \endhead
%-
  \multicolumn{3}{@{}r}{\small{\emph{Fortsetzung nächste Seite}}\ldots} \\
  \endfoot
  \endlastfoot
%% -----------------------------------------------
    \emph{Zahlenmengen:} \\
    \verb| \N | & $ \to $ & $ \N $ \\ 
    \verb| \Z | & $ \to $ & $ \Z $ \\
    \verb| \Q | & $ \to $ & $ \Q $ \\
    \verb| \R | & $ \to $ & $ \R $ \\
    \verb| \C | & $ \to $ & $ \C $ \\
    \verb| \K | & $ \to $ & $ \K $ \\
	\midrule
	\emph{Integral:} \\
    \verb| \ds | & $ \to $ & $ \ds $ \\
    \verb| \dt | & $ \to $ & $ \dt $ \\
    \verb| \dx | & $ \to $ & $ \dx $ \\
    \verb| \diff{\mu} | & $ \to $ & $ \diff{\mu} $ \\
    \verb| \d{\mu} | & $ \to $ & $ \diff{\mu} $ \\
	\midrule
	\emph{Variable:} \\
    \verb| \phi | & $ \to $ & $ \phi $ \\
    \verb| \epsilon  | & $ \to $ & $ \epsilon $ \\
    \verb| \rho | & $ \to $ & $ \rho $ \\
	\verb| \theta | & $ \to $ & $ \theta $ \\
	\verb| \leq | & $ \to $ & $ \leq $ \\
	\verb| \geq | & $ \to $ & $ \geq $ \\
	\midrule
	\emph{Sonstiges: } \\
	\verb| \abs{x} | & $ \to $ & $ \abs{x} $ \\
	\verb| \abs{} | & $ \to $ & $ \abs{} $ \\
	\verb| \norm{x} | & $ \to $ & $ \norm{x} $ \\
	\verb| \norm{} | & $ \to $ & $ \norm{} $ \\
	\verb| \norm*{\sum} | & $ \to $ & $ \norm*{\sum} $ \\
	\midrule
	\verb| \interval{a,b}}| & $ \to $ & $ \interval{a,b}  $ \\
	\verb| \rointerval{a,b}}| & $ \to $ & $ \rointerval{a,b}  $ \\
	\verb| \lointerval{a,b}}| & $ \to $ & $ \lointerval{a,b}  $ \\
	\verb| \ointerval{a,b}}| & $ \to $ & $ \ointerval{a,b}  $ \\
% -----------------------------------------------
    \bottomrule
  \end{longtable}
\end{center}
%%
\begin{remark}
Kleine Anmerkung: Mittels der Eingabe der Sternvariante von \cs{abs} oder \cs{norm} sich die Größe der Begrenzungen an den folgenden Text an.
Etwa
%
\begin{tcblisting}{}
%
\[
	\abs*{ \frac{1}{\frac{a}{b+c}} }
\]
%
oder
%
\[
	\norm*{ \frac{1}{\frac{a}{b+c}} }
\]
%
\end{tcblisting}
\end{remark}
%%
Für die mathematischen Umgebungen stehen folgende Abkürzungen zur Verfügung:
\begin{center}
  \begin{longtable}{@{} lcl @{}}
  \caption{Mathematische Umgebungen}\label{tab:ma-umgebungen} \\
%-
  \toprule
  Die Eingabe & ergibt die  & folgende Umgebung \\ 
  \toprule
  \endfirsthead  
%-
  \multicolumn{3}{@{}l}{\small\ldots\emph{Fortsetzung}} \\
  \toprule
  Der Schlüssel \emph{key} & ergibt die  & folgende Umgebung \\ 
  \toprule
  \endhead
%-
  \multicolumn{3}{@{}r}{\small{\emph{Fortsetzung nächste Seite}}\ldots} \\
  \endfoot
  \endlastfoot
% ------------------------------------------------------------------------------
    \texttt{theorem} oder \texttt{thm} & $ \to $ & Theorem \\ 
    \texttt{prop}, \texttt{proposition} oder \texttt{satz} & $ \to $ & Satz \\
    \texttt{cor}, \texttt{corollary} oder \texttt{korollar} & $ \to $ & Korollar \\
    \texttt{lem} oder \texttt{lemma} & $ \to $ & Lemma \\
	\texttt{defn} oder \texttt{definition} & $ \to $ & Definition \\
	\texttt{examp}, \texttt{beispiel} oder \texttt{example} & $ \to $ & Beispiel \\
	\texttt{rem} oder \texttt{note} & $ \to $ & Anmerkung \\
% ------------------------------------------------------------------------------
    \bottomrule
  \end{longtable}
\end{center}
%
\begin{tcblisting}{title= Ein Beispiel}
\begin{theorem}
\ldots
\end{theorem}
\end{tcblisting}
%
\subsection{Wo bekommt man Hilfe?}
%
Bei der Nutzung des Systems wird man immer wieder auf Probleme stoßen.
Vieles wird man selbst lösen können, da es sich meistens um Eingabefehler handelt.
Kommt man aber nicht weiter, so gibt es verschiedene Webseiten, auf denen man Hilfe bekommt

\begin{itemize}
\item
Für alle möglichen Fragen, Installation \ua: \url{http://projekte.dante.de/DanteFAQ/WebHome}
\item
Für technische Probleme: \url{https://tex.stackexchange.com}

\end{itemize}

Und dann kann noch Google nutzen -- aber Achtung: Nicht alles, was man dann findet ist sinnvoll.
Hier sollte man auf das Datum der Frage achten. 
