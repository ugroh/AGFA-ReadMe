%% --
%% ./content/Abschnitt0.tex = Abschnitt0 Kleine Einleitung
%% -- Stand 2023/01/29
\dictum[Paula Fox]{Ein guter Roman beginnt mit einer kleinen Frage und endet mit einer größeren.%
	\footnote{Gilt auch für eine mathematische Arbeit}}
%%
\thispagestyle{empty}
\section{Einleitung}
%Um den Einstieg in \LaTeX{} etwas zu erleichtern, sind im Folgenden einige Punkte zusammengestellt, die für die ersten Schritte nützlich sind. 
%Alles, was blau markiert ist, ist mit Links auf weiterführende Informationen hinterlegt.
%Des Weiteren habe ich in einige \LaTeX{}-Tipps separat erstellt, die einiges vertiefen, was ich hier nur kurz anspreche -- diese finden sich auf ILIAS.
% 
\subsection{Was ist \LaTeX{}}
%\subsubsection{}
\LaTeX{} ist eine \href{https://de.wikipedia.org/wiki/Auszeichnungssprache}{Markup-Sprache}, die auf dem \href{https://de.wikipedia.org/wiki/Satz_(Druck)}{Textsatzsystem} \TeX{} basiert und ist, vor allem im naturwissenschaftlichen Bereich, zu einem \emph{de facto} Standard geworden.
Im Gegensatz zu den \href{https://de.wikipedia.org/wiki/WYSIWYG}{\emph{What You See is What You Get}} Systemen wie etwa Word, wird hier mittels Steuerelemente die Gestalt (Layout) des Dokuments festgelegt 
-- \href{https://de.wikipedia.org/wiki/WYSIWYM}{\emph{What You See is What You Mean}}.
Der Nutzer kann sich somit ganz auf den \emph{Inhalt} seiner Arbeit konzentrieren.
Dies ist zwar am Anfang etwas aufwendiger zu erlernen ist, aber es ist dadurch flexibler und besser auf die eigenen Bedürfnisse anpassbar.%
\footnote{Nebenbei: Word lernt man auch nicht über Nacht und für mathematischen Text ist dieses System weitestgehend unbrauchbar.}

Zur Geschichte von \TeX{} und die Gründe, warum es \href{https://de.wikipedia.org/wiki/Donald_E._Knuth}{Donald Knuth} vor über 50 Jahren geschaffen hat, findet man in seinem Buch \enquote{Digital Typography} \cite{knuth:digital} oder in \textcite{beeton:math}.
\href{https://de.wikipedia.org/wiki/LaTeX}{\LaTeX{}} selbst ist ein Makropaket, das auf \href{https://de.wikipedia.org/wiki/Leslie_Lamport}{L. Lamport} zurückgeht, der dieses um 1983 herum entwickelt hat -- siehe hierzu seine Erläuterungen in \cite{lamport:dmv}.

Eine gute Referenz für einen ersten Einstieg ist \textcite{lshort-german}, da sich hier alles Wesentliche zur Nutzung von \LaTeX{} findet.
Zu beachten ist aber, dass sich diese Einleitung auf \LaTeX{} des Jahres 2001 bezieht -- zwischenzeitlich ist einiges passiert, was die Eingabe erleichtert.
Dies betrifft vor allem die Eingabe eines Textes mit Zeichensätzen außerhalb des anglo-amerikanischen Sprachraums.
Trotzdem -- bitte unbedingt nutzen.

Als Literatur ist das RRZN-Handbuch \emph{ \LaTeX{}-—Einführung in das Textsatzsystem} zu empfehlen, das man leider über das hiesige Rechenzentrum der Universität Tübingen nicht beziehen kann.%
\footnote{Bei Interesse bin ich gern bereit eine Sammelbestellung zu initiieren.} 
Die Bücher von Herbert Voß -- siehe hierzu \url{https://www.dante.de/dante-e-v/literatur/} -- sind für alle empfohlen, die sich intensiver mit \LaTeX\ beschäftigen wollen.
Zu empfehlen ist \textcite{voss:2012a} \enquote{\citetitle{voss:2012a}} als Begleitlektüre.%
\footnote{Gut ist auch die Begleitdokumentation von Overleaf, das man sich via \texttt{Help} anzeigen lassen kann}

Für deutsche Texte sind die Dokumentenklassen, die auf KOMA-Script beruhen \textcite{kohm:2020}, da hier die Gegebenheiten bei uns berücksichtigt sind -- die Klassen von KOMA-Script werden in den Musterdateien genutzt.

\subsection{Wie startet man}
Zum Start bieten sich zwei Alternativen an:
%
\begin{myitemize}[nosep]
	\item 
	Eine lokale Installation auf seinen eigenen Laptop oder PC -- dies ist meine Empfehlung, wenn dies möglich ist.
	
	\item
	Die Nutzung des Onlineangebots Overleaf \url{https://de.overleaf.com} -- dazu das zip-File in Overleaf installieren
	(siehe hierzu die Anleitung zu der Vorlage).
	
\end{myitemize}
%
Zur lokalen Installation nutzt man die über die \TeX-Users Group (TUG) via \href{https://tug.org/texlive/}{TeX Live} zur Verfügung gestellt wird. 
Dies betrifft Systeme mit Windows oder Ubuntu und dieses wird gepflegt, \dh es gibt jährlich ein Update.

Wer zu den Glücklichen gehört, die einen Mac nutzen (mit OS X), für die steht eine auf TeX Live basierendes System zur Verfügung (\url{https://tug.org/mactex/}).
Dieses System beinhaltet auch einen sehr guten Editor, \href{https://pages.uoregon.edu/koch/texshop/}{TeXshop}, ein Verwaltungsprogramm für die Literatur, \href{https://bibdesk.sourceforge.io}{BibDesk} und eine Reihe von Tools, die einem das Erstellen von \TeX{}-Dokumenten erleichtert..
Einführung zu diesen Programmen findet man auch auf YouTube.

Wie erwähnt, arbeitet \TeX{} mit Steuerzeichen die \enquote{sagen}, was gemacht werden soll.
Man kann daher dieses System mit einer Programmiersprache vergleichen und es ist somit anfällig gegen Fehler bei der Eingabe dieser Steuerzeichen. 
Eine einfache Regel für den Anfang: Sparsam sein bei der Verwendung von Steuerzeichen und nicht verzweifeln im Fehlerfall.
Meistens stimmen die Klammern, speziell im Mathematikmodus, paarweise nicht!

\subsection{Bitte beachten}
%\subsubsection{}
%
Ein \enquote{\LaTeX{}-Sourcefile} besteht immer aus drei Teilen:

\begin{myenumerate} 
	\item 
	\emph{Aus der Präambel:} Dies ist alles zwischen \verb|\documentclass[..| und \par
	\verb|\begin{document}|.
	Hier finden sich (in der Regel) eigene Definitionen, der Aufruf von speziellen Paketen, die man als Ergänzung nutzt etc.
	Dazu einfach die Musterdatei und die Referenzen ansehen.
	
	\item
	\emph{Aus dem Hauptteil: } Nach dem \verb!\begin{document}! startet der Teil, der den Inhalt darstellt.
	Dieser wird entsprechend untergliedert und die einzelne Abschnitte mit Überschriften versehen.
	Wie man dieses machen kann -- siehe das Muster.
		
	\item
	\emph{Aus dem Schluss: } Dieser startet mit der Ausgabe der Literatur mittels der Umgebung für das Literaturverzeichnis -- \verb!\printbibiography! -- beinhaltet eventuell den Index \etc und endet mit \verb!\end{document}!.
	
	\item
	Nach der Umwandlung und wenn alles richtig ist, hat man ein PDF-Dokument mit einem optisch ansprechenden \href{https://de.wikipedia.org/wiki/Layout}{Layout}.
\end{myenumerate}
%
Ich habe ein kleines \LaTeX{}-File vorbereitet, das man für die ersten Gehversuche und die Erstellung der Ausarbeitung für die Hausarbeit nutzen kann.
Diese Vorlage ist aber für eine Bachelor- oder Masterarbeit nicht ausreichend, aber man kann diese als Basis nehmen und entsprechend \enquote{ausbauen}.%
\footnote{Ein Template für Bachelor- oder Masterarbeiten für AGFA findet sich auf GitHub unter \url{https://github.com/ugroh/AGFA-Master}}


\subsection{Einige Tipps }
Noch einige Tipps: 

\begin{myenumerate}
	\item
	Starte jeden neuen Satz auf einer neuen Zeile.
	Dies macht alles übersichtlicher und hilft, wenn man Fehler im Code sucht.
	Es ist \TeX{} kein System, mit dem man \href{https://de.wikipedia.org/wiki/Fließtext}{Fließtext} schreiben sollte.
	
	\item
	Bitte nicht \verb!\\! oder \verb!\newline! zu verwenden, um einen neuen Absatz zu erhalten.
	Will man einen neunen Absatz haben, so macht man dies mittels einer Leerzeile im laufenden Text.
	Den Rest -- Trennung nach den deutschen Regeln \ua -- macht dann das Programm.
	
	\item
	Wer sich unsicher ist, ob die Rechtschreibung oder die Zeichensetzung stimmt, der findet unter 
	\href{https://grammis.ids-mannheim.de}{https://grammis.ids-mannheim.de} Hilfe.
	Und es gibt sogar Programme, die einem bei der Erstellung von Texten helfen, etwa \href{https://languagetool.org/de}{LanguageTool}.
	Ganz zu schweigen von den KI-Tools wie \href{https://openai.com/blog/chatgpt/}{chatGTP }
	
	\item
	Nutze \textcite{lshort-german} für die ersten \enquote{Gehversuche} in \LaTeX\@.
	Aber bitte beachten: Es verwendet die Dokumentenklassen von \LaTeX{} als Beispiele, die aber nicht für die Eigenheiten der deutschen Sprache geeignet sind.
	Besser ist es mit KOMA-Script zu arbeiten, wie es in den Vorlagen gemacht wurde (\textcite{kohm:2020}).
	
	\item
	Es ist nicht mehr erforderlich, etwa Umlaute, durch spezielle Befehle einzugeben, wie es früher notwendig war.
	Wer seinen Editor korrekt auf UTF-8 eingestellt hat, kann einfach schreiben.
	
	\item
	Es gibt einige typographische Regeln, sowohl für die Eingabe eines Textes als auch für die Mathematik.
	Mehr dazu findet man unter \href{http://menetekel.e-technik.fh-muenchen.de/skripten/LaTeX/typokurz.pdf}{typokurz}, auf \href{https://www.typolexikon.de}{dem TypoLexikon Online} und unter \textcite{nadler:formelsatz} nützliche Tipps.
	
	\item
	Für den Einstieg in die Mathematik mit \LaTeX{} empfehle ich den AMS-Guide \cite{short-math-guide}.
	Weiteres findet man dann etwa in \textcite{graetzer:2007} oder in \textcite{voss:2012b}.	
	
	\item
	Zu empfehlen ist auch das Interview mit Leslie Lamport und seine drei Empfehlungen zu verinnerlichen (\cite{lamport:dmv})
	
	\item
	Alle diejenigen, die schon \LaTeX{} nutzen: Mal in \textcite{l2tabu} reinschauen.
	Dort finden sich alle Sünden, die man bei der Nutzung des Systems nicht machen soll.
	
		
\end{myenumerate}