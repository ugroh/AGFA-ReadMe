% !TEX root = ../agfa-readme.tex
%% %%%%%%%%%%%%%%%%%%%%%%%%%%%%%%%
%% Zusammenfassung zu ../agfa-readme.tex
%% Stand: 2022/10/10
%% %%%%%%%%%%%%%%%%%%%%%%%%%%%%%%%
\thispagestyle{empty}
\dictum[G.\,W.\,F. Hegel]{Die Wahrheit einer Absicht ist die Tat.}
\addsec{Was beinhaltet die Vorlage}
Dies ist eine Übersicht zu der Vorlage, die ich für AGFA erstellt habe und die den Zweck hat, die Anfertigung einer Arbeit, sei es nun eine Bachelorarbeit, eine Masterarbeit oder eine Dissertation, zu unterstützen.
Für die Erstellung habe ich im Wesentlichen genutzt: 
%
\begin{myitemize}[nosep]
\item
M. Kohm, \emph{KOMA-Script}, \cite{kohm:2020}   

\item
H. Voß, \emph{Erstellung einer wissenschaftlichen Arbeit mit \LaTeX{}}, \cite{voss:2021}.

\item
Als Alternative mit den wesentlichen \LaTeX\ Informationen: \textcite{schlosser:2016}.

\end{myitemize}
%
Alles, was ich in den Vorlagen verwende, ist in diesem ReadMe ausführlich beschrieben.
Eine Zusammenfassung zu KOMA-Script und dessen Möglichkeiten gibt es auch noch \href{https://ctan.org/pkg/latex-refsheet}{das \LaTeX\ Reference Sheet} -- bitte nutzen.
Unabhängig davon empfehle ich die Kurzeinführung \textcite{lshort-german} in \LaTeX{}, auch wenn sich diese Anleitung auf eine ältere \LaTeX{}-Version stützt und manches heute nicht mehr erforderlich ist.

Was das Schreiben eines mathematischen Textes betrifft, so ist
%%
\begin{myitemize}[nosep]
\item
\href{https://sites.math.washington.edu/~lind/Resources/Halmos.pdf}{How to Write Mathematics} von P. Halmos 
%%
\end{myitemize}
%
Pflichtlektüre für jeden Mathematiker.
Als Ergänzung dazu bitte auch den \href{https://www.ams.org/notices/200709/tx070901136p.pdf}{Nachruf auf P. Halmos} lesen.
Auf YouTube findet sich auch ein \href{https://www.youtube.com/watch?v=Cy_1JgYfKmE}{Video} dazu.
Dies ist ein Vortrag im Rahmen einer Vorlesungsreihe von D. Knuth zum Thema \enquote{Mathematical Writing} -- es lohnt sich, dieses anzusehen.
Und wer auch noch wissen will, warum es \TeX{} und damit \LaTeX{} gibt: siehe \textcite{knuth:digital}, \textcite{beeton:math} und \textcite{lamport:dmv}.

Des weiteren gehe ich davon aus, dass jeder ein aktuelles \TeX{}-System auf seinen Rechner hat. 
Dieses findet man für Windows oder Linux unter \href{https://tug.org/texlive/}{https://tug.org/texlive/}.
Für Mac OS X gibt es unter \href{https://tug.org/mactex/}{https://tug.org/mactex/} das aktuelle System mit Editor und einem Verwaltungsprogramm für die Literatur.
Eine Alternative ist das System \href{https://www.overleaf.com}{Overleaf}, auf das man mittels eines Browsers online zugreifen kann.
Die entsprechenden Anleitungen und weitergehenden Informationen finden sich auf den angegeben Webseiten.

Die Vorlagen selbst finden sich auf GitHub 
\href{https://github.com/ugroh/AGFA-Master}{https://github.com/ugroh/AGFA-Master} und kann als ZIP-File herunter geladen werden; siehe hierzu das \texttt{ReadMe.md} File auf Github. 

\mytodo{Das ReadMe für AGFA-Sem und dieses hier zusammenfassen}






